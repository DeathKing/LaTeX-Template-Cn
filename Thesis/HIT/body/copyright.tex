\defaultfont

%%%%%%%%%%%%%%%%%%%%%%%%%%%%%%%%%%%%%%%%%%%%%%%%%%%%%%%%%%%%%%%%%%%%%%%%%%%%%
\BiChapter{��Ȩ����}{Copyright Statement}

��ģ����ѭ~GPL~Э�顣���������������г���\\
UFO\\
cucme\\
Stanley\\
TeX\\
nebula\\
�����������������ǡ�
